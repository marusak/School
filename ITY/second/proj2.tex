\documentclass[a4paper, 11pt, twocolumn] {article}

\usepackage{times}
\usepackage[czech]{babel}
\usepackage[utf8]{inputenc}
\usepackage[left=1.5cm,text={18cm, 25cm},top=2.5cm]{geometry}

\usepackage{amsmath, amsthm, amssymb}


\begin{document}
\begin{titlepage}
\begin{center}

\newtheorem{definicia}{Definice}[section]
\newtheorem{algoritmus}[definicia]{Algoritmus}
\newtheorem{veta}{Věta}



{\Huge \textsc{Fakulta informačních technologií\\
Vysoké učení technické v~Brně}}\\

\vspace{\stretch{0.382}}

{\LARGE
	Typografie a publikování\,--\,2. projekt\\
	Sazba dokumentů a matematických výrazů\\
}

\vspace{\stretch{0.382}}
\end{center}

{\LARGE 2017    \hfill     Matej Marušák}
\end{titlepage}


\section*{Úvod}
V~této úloze si vyzkoušíme sazbu titulní strany, matematických vzorců, prostředí a dalších textových struktur obvyklých pro technicky zaměřené texty například rovnice (\ref{r1}) nebo definice \ref{d11} na straně \pageref{d11}.

Na titulní straně je využito sázení nadpisu podle optického středu s~využitím zlatého řezu. Tento postup byl probírán na přednášce.


\section{Matematický text}

Nejprve se podíváme na sázení matematických symbolů a výrazů v~plynulém textu. Pro množinu $V$ označuje card($V$) kardinalitu $V$.
Pro množinu $V$ reprezentuje $V^*$ volný monoid generovaný množinou $V$ s~operací konkatenace.
Prvek identity ve volném monoidu $V^*$ značíme symbolem $\varepsilon$.
Nechť $V^+ = V^* - \{\varepsilon\}$. Algebraicky je tedy $V^+$ volná pologrupa generovaná množinou $V$ s~operací konkatenace.
Konečnou neprázdnou množinu $V$ nazvěme $abeceda$.
Pro $w \in  V^*$ označuje $|w|$ délku řetězce $w$. Pro $W \subseteq V$ označuje occur($w,W$) počet výskytů symbolů z~$W$ v~řetězci $w$ a sym($w, i$) určuje $i$-tý symbol řetězce $w$; například sym($abcs, 3$) $=c$.

Nyní zkusíme sazbu definic a vět s~využitím balíku \texttt{amsthm}.

\begin{definicia} \label{d11} Bezkontextová gramatika \textup{je čtveřice $G=(V,T,P,S)$, kde $V$ je totální abeceda,
$T \subseteq V$ je abeceda terminálů, $S \in (V~- T)$ je startující symbol a $P$  je konečná množina} pravidel
\textup{tvaru $q \colon A~\rightarrow \alpha$, kde $A \in  (V~- T), \alpha \in V^*$ a $q$ je návěští tohoto pravidla. Nechť $N = V~- T $ značí abecedu neterminálů.
Po\-kud $q \colon A~\rightarrow \alpha \in P$,$ \gamma, \delta \in V^*$, $G$ provádí derivační krok z~$\gamma A~\delta$ do $\gamma \alpha \delta$ podle pravidla $q \colon A~\rightarrow \alpha$, symbolicky píšeme 
 $\gamma A~\delta \Rightarrow \gamma \alpha \delta [q \colon A~\rightarrow \alpha]$ nebo zjednodušeně $\gamma A~\delta \Rightarrow \gamma \alpha \delta$. Standardním způsobem definujeme $\Rightarrow^m$, kde $m \geq 0$ . Dále definujeme 
tranzitivní uzávěr $\Rightarrow^m$ a tranzitivně-reflexivní uzávěr $\Rightarrow^*$ .}
\end{definicia}

Algoritmus můžeme uvádět podobně jako definice textově, nebo využít pseudokódu vysázeného ve vhodném prostředí (například \texttt{algorithm2e}).

\begin{algoritmus} \label{a1} Algoritmus pro ověření bezkontextovosti gramatiky. Mějme gramatiku $G = (N, T, P, S)$.
\begin{enumerate}
 \item \label{krok1} Pro každé pravidlo $p \in P$ proveď test, zda $p$ na levé straně obsahuje právě jeden symbol z~$N$ .
 \item Po\-kud všechna pravidla splňují podmínku z~kroku \ref{krok1}, tak je gramatika $G$ bezkontextová.
\end{enumerate}
\end{algoritmus}

\begin{definicia}
\textup{Jazyk definovaný gramatikou $G$ definujeme jako $L(G) = \{ \in T^*|S \Rightarrow^* w\}$.}
\end{definicia}

\subsection{Podsekce obsahující větu}

\begin{definicia} \textup{Nechť $L$ je libovolný jazyk. $L$ je} bezkontextový jazyk \textup{, když a jen když $L=L(G)$, kde G je libovolná bezkontextová gramatika.}
\end{definicia}

\begin{definicia}
\textup{Množinu $\mathcal{L}_{CF} = \{L|L$ je bezkontextový jazyk $\}$ nazýváme} třídou bezkontextových jazyků.
\end{definicia}

\begin{veta}
\label{v1}
Nechť $L_{abc} = \{a^nb^nc^n|n \geq 0\}$. Platí, že $L_{abc} \notin  \mathcal{L}_{CF} $.
\end{veta}

\begin{proof}
Důkaz se provede pomocí Pumping lemma pro bezkontextové jazyky, kdy ukážeme, že není možné, aby platilo, což bude implikovat pravdivost věty \ref{v1} .
\end{proof}


\section{Rovnice a odkazy}

Složitější matematické formulace sázíme mimo plynulý text. Lze umístit několik výrazů na jeden řádek, ale pak je třeba tyto vhodně oddělit, například příkazem \verb|\quad|. 

$$\sqrt[x^2]{y^3_0} \quad \mathbb{N} = \{0,1,2,\dots\} \quad x^{y^y} \neq x^{yy} \quad z_{i_j} \not\equiv z_{ij} $$

V~rovnici (\ref{r1}) jsou využity tři typy závorek s~různou explicitně definovanou velikostí.

\begin{eqnarray}
\bigg\{\Big[\big(a+b\big)*c\Big]^d+1\bigg\} & =  & x \label{r1}\\ 
\lim\limits_{x \to \infty}\frac{\sin^2\,{x} + \cos^2\,{x}}{4} & = & y \nonumber
\end{eqnarray}


V~této větě vidíme, jak vypadá implicitní vysázení limity $\lim_{n \to \infty}{f(n)}$ v~normálním odstavci textu. Podobně je to i s~dalšími symboly jako $\sum_{1}^{n}$ či $\bigcup_{A \in \mathcal{B}}$ . V~případě vzorce $\lim\limits_{x \to \infty}{\frac{\sin\,x}{x}}  = 1 $ jsme si vynutili méně úspornou sazbu příkazem \verb|\limits|.

\begin{eqnarray}
\int\limits_a^b f(x)\,\mathrm{d}x & = & -\int_b^a f(x)\, \mathrm{d}x\\
\Big(\sqrt[5]{x^4}\Big)' = \Big(x^{\frac{4}{5}}\Big)' & = & \frac{4}{5}x^{-\frac{1}{5}} = \frac{4}{5\sqrt[5]{x}}\\
\overline{\overline{A \vee B}} & = & \overline{\overline{A} \vee \overline{B}}
\end{eqnarray}


\section{Matice}

Pro sázení matic se velmi často používá prostředí \texttt{array} a závorky (\verb|\left|, \verb|\right|). 

$$\left(
\begin{array}{ccc}
a + b & b - a \\
\widehat{\xi + \omega} & \hat{\pi}\\
\vec{a}  & \overleftrightarrow{AC}\\
0 & \beta
\end{array}
\right)
$$

$$
\mathbf{A} = 
\left\|
\begin{array}{cccc}
a_{11} & a_{12} & \dots & a_{1n}\\
a_{21} & a_{22} & \dots & a_{2n}\\
\vdots & \vdots & \ddots & \vdots\\
a_{m1} & a_{m2} & \dots & a_{mn}
\end{array}
\right\|
$$

$$
\left|
\begin{array}{lcr}
t & u\\
v~& w\\
\end{array}
\right|
= tw -uv
$$

Prostředí \texttt{array} lze úspěšně využít i jinde.

$$
\binom{n}{k} =
\left\{
\begin{array}{ll}
\frac{n!}{k!(n - k)!} & \text{pro } 0 \leq k~\leq n\\
0 & \text{pro } k~< 0 \text{ nebo } k~> n
\end{array} \right.
$$

\section{Závěrem}

V~případě, že budete potřebovat vyjádřit matematickou konstrukci nebo symbol a nebude se Vám dařit jej nalézt v~samotném \LaTeX u, doporučuji prostudovat možnosti balíku maker \AmS -\LaTeX.
Analogická poučka platí obecně pro jakoukoli konstrukci v~\TeX u.

\end{document}