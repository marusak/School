\documentclass[a4paper, 11pt] {article}

\usepackage{times}
\usepackage[czech]{babel}
\usepackage[utf8]{inputenc}
\usepackage[left=1.5cm,text={18cm, 25cm},top=2.5cm]{geometry}

\usepackage{amsmath, amsthm, amssymb}
\bibliographystyle{czplain}


\begin{document}
\begin{titlepage}
\begin{center}

\newtheorem{definicia}{Definice}[section]
\newtheorem{algoritmus}[definicia]{Algoritmus}
\newtheorem{veta}{Věta}



{\Huge \textsc{Fakulta informačních technologií\\
Vysoké učení technické v~Brně}}\\

\vspace{\stretch{0.382}}

{\LARGE
	Typografie a publikování\,--\,4. projekt\\
	Bibliografické citácie\\
}

\vspace{\stretch{0.618}}
\end{center}

{\LARGE 2017    \hfill     Matej Marušák}
\end{titlepage}

\section{Úvod}
Typografia je umenie a technika aranžovania písma. Rovnako ako rečnícke schopnosti osoby, tak aj kvalita písomného prejavu dokáže ovplyvniť ako ľudia reaguju na našu správu\cite{pow_typo}. Aby sme mohli ovládať náš písomný prejav aj pri sadzbe dokumentov na počítači, bol vyvinutý nástroj \TeX.

\section{Systém \TeX}
Voľne šíriteľný nástroj \TeX ktorého autorom je Donald E. Knuth zo Stanfortskej univerzity je vyvrcholením snahy o dokonalú počítačovú sadzu odborných textov. Pôvod slova \TeX vysvetľuje sám autor v \cite{knuth} nasledovne:
\begin{quotation}

Anglické slová ako \uv{technology} pochádzajú z grétskeho základu začínajúceho s písmenami $\tau\epsilon\chi...$; a to isté grétske slovo znamená umenie ako aj technológie. Odtiaľ meno \TeX, ktoré je veľkými písmenami  $\tau\epsilon\chi$.
\end{quotation}

Systém obsahuje viac ako 300 príkazov, ktorými sa okrem iného dajú upravovať a vytvárať nové vlastnosti. Vďaka tomu boli vytvorené nadstavby, ktoré umožňujú jednoduchší a prirodzenejší zápis textu. Tento proces mi môžme predstaviť ako tvorba vyšších programovacích jazykov, ktoré obsahujú zložitejšie funkcie, avšak tieto funkcie sú zapísané v jazyku nižšom\cite{rybicka}.

Tento nástroj je natoľko populárny, že vznikajú nespočetné webové stránky a fóra ale i časopisy. Medzi významné, i keď dnes už nevychádzajúce časopisy patrí PracTex\cite{practex} alebo domáca Typografia	\cite{typografia}.

\section{Nadstavba \LaTeX}
Najznámejšou nadstavbou je pravdepodobne \LaTeX, ktorý bol vytvorený Leslie Lamportom. Táto nadstavba sa prevažne využíva na tvorbu stredných až veľkých technických a vedeckých prác avšak je ju možné využiť takmer na tvorbu akéhokoľvek dokumentu. Myšlienkou \LaTeX-u je podporovať autorov k zameraniu sa pri tvorbe na obsah a nie na vzhľad písaného dokumentu \cite{latex_webpage}. 

\LaTeX sa stal natoľko populárnou nadstavbou, že často je práve označenie \LaTeX používané ako synonymum pre \TeX. Jednoduché a výstižné vysvetlenie rozdielu je možné nájsť v podsekcii \ref{tex_vs_latex}.

\subsection{\TeX verzus \LaTeX}
\label{tex_vs_latex}
Užívatelia \LaTeX-u často nepoznajú rozdiel medzi \TeX-om a \LaTeX-om. \TeX je program na sadzbu textu, zatiaľ čo \LaTeX je sada makier pre \TeX. Na tieto makrá sa môžme pozerať aj ako na knižnicu príkazov\cite{mks}.


\section{Nástroje na tvorbu dokumentu}
Nástroje môžme rozdeliť na offline a online podľa potreby pripojenia k internetu. Porovnanie editorov je možné nájsť na \cite{prekladace_wiki}.	

\subsection{Offline nástroje}
Bežným prístupom je vytváranie dokumentov lokálne na počítači. Avšak programy pre tvorbu nie sú príliš rozšírené a známe ako je to napríklad s textovým procesorom Word. Samotný editor nie je dostačujúci a je potrebné nainštalovať aj daľšie súčasti a rozšírenia. Takáto inštalácia a konfigurácia nie je pre bežného užívateľa triviálna záležitosť\cite{sokol}.


\subsection{Online nástroje}
V poslednej dobe vznikajú viaceré nástroje na tvorbu \LaTeX dokumentov priamo v internetovom prehliadači. Umožňujú zdielanú tvorbu, či už jedným autorom na viacerých počítačoch, tak aj viacerých autorov spolu. Takéto nástroje umožňujú jednoduchšiu tvorbu dokumentov v \LaTeX-u, zbavujú hlavne začínajúcich autorov frustrácie zo zle nastaveného prostredia a chýbajúcich súčasti a umožnujú kolektívnu spoluprácu\cite{online_tex}. Medzi známe online editory dnes patri ShareLatex\footnote{Dostupné na: https://www.sharelatex.com/} alebo Overleaf\footnote{Dostupné na: https://www.overleaf.com/}.

\section{Čeština v \LaTeX-u}
Podpora češtiny v \LaTeX-u vyžaduje 3 veci:
\begin{itemize}
\item Kódovanie - Je vhodné vybrať UTF8. Na to môžme použiť príkaz \begin{verbatim}\usepackage[utf8]{inputenc}
\end{verbatim}
\item České odlišnosti - Čeština sa líši vo viacerých ohľadoch voči angličtine. Pridaním 
\begin{verbatim}
\usepackage[czech]{babel}
\end{verbatim}
tieto zmeny zapneme. Okrem toho obsahuje napríklad aj spôsob delenia slov.
\item Bibliografia podľa ČSN - Je vhodné použiť špeciálne štýly pri tvorbe bibliografie, aby odpovedala aktuálnym normám. Jeden taký štýl bol vytvorený v \cite{czplain_bp}
\end{itemize} 



\newpage
\renewcommand{\refname}{Referencie}
\bibliography{bibliografia}


\end{document}