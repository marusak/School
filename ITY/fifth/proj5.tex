\documentclass[slovak]{beamer}
\usepackage[slovak]{babel}
\usepackage[utf8]{inputenc}
\usepackage{hyperref}

\usetheme{Madrid}

\title{Tvorba prezentácií v \LaTeX\,u}
\subtitle{Za využitia triedy \texttt{beamer}}
\author{M.Marušák}
\institute[Vysoké učení technické]
{
  Vysoké učení technické v Brne
}
\date{ITY, 2017}

\begin{document}

\begin{frame}
  \titlepage
\end{frame}

\begin{frame}{Úvod}
  \tableofcontents
\end{frame}

\section{Základná štruktúra}

\begin{frame}[fragile]{Základná štruktúra}
\begin{verbatim}
\documentclass[slovak]{beamer}
\usepackage[slovak]{babel}
\usepackage[utf8]{inputenc}

\usetheme{Madrid}

\title{Tvorba prezentácií v \LaTeX\,e}
\subtitle{Za využitia triedy \texttt{beamer}}
\author{M.Marušák}
\date{ITY, 2017}

\begin{document}
    Presentation Body
\end{document}
\end{verbatim}
\end{frame}


\section{Témy}
\begin{frame}{Témy}

\begin{itemize}
\item existuje mnoho tém, ktoré určujú celkový vzhľad prezentácie
\item zoznam s ukážkami: \url{http://deic.uab.es/~iblanes/beamer\_gallery/index\_by\_theme.html}
\end{itemize}

\begin{minipage}{0.30\textwidth}%
\begin{block}

\begin{itemize}
\item AnnArbor
\item Antibes
\item Bergen
\item Berkeley
\item Berlin
\item Boadilla
\item boxes
\item CambridgeUS
\item Copenhagen
\end{itemize}
\end{block}
\end{minipage}
\hspace{0.5cm}
\begin{minipage}{0.30\textwidth}%
\begin{block}

\begin{itemize}
\item Darmstadt
\item default
\item Frankfurt
\item Goettingen
\item Hannover
\item Ilmenau
\item JuanLesPins
\item Luebeck
\item Madrid
\end{itemize}

\end{block}

\end{minipage}
\hspace{0.5cm}
\begin{minipage}{0.25\textwidth}%
\begin{block}

\begin{itemize}
\item Malmoe
\item Marburg
\item Montpellier
\item PaloAlto
\item Pittsburgh
\item Rochester
\item Singapore
\item Szeged
\item Warsaw
\end{itemize}

\end{block}

\end{minipage}

\end{frame}

\section{Základné údaje}
\begin{frame}{Základné údaje}
\begin{block}{Základné údaje}

\begin{itemize}
\item title
\item subtitle
\item author
\item date
\end{itemize}
\end{block}
\begin{itemize}
\item tvorí sa z nich titulná snímka
\item zobrazujú sa na každom slide
\item ďalšie údaje - email, institute...
\end{itemize}
\end{frame}

\section{Tvorba slidov}
\begin{frame}[fragile]{Tvorba slidov}
\begin{block}{Štruktúra slidu}
\verb|\section{Tvorba slidov}|\\
\verb|\begin{frame}{Tvorba slidov}|
\setlength{\unitlength}{1cm}
\begin{picture}(0,0)
\put(-0.65,0.3){\vector(-1,1){3}}
\end{picture}
\\
\verb|    ... obsah slidu ...|\\
\verb|\end{frame}|
\end{block}
\begin{itemize}
\item sekcia môže obsahovať viacero slidov
\item slide je uzatvorený v \uv{frame}
\end{itemize}

\end{frame}

\section{Tipy a triky}
\begin{frame}[fragile]{Bloky}
\begin{itemize}
\item názov/bez názvu
\item 3 typy - \texttt{block} | \texttt{alertblock} | \texttt{exampleblock}
\end{itemize}

\begin{block}{Klasický blok}
\verb|\begin{block}{Klasický blok}|\\
... obsah bloku ...\\
\verb|\end{block}|\\
\end{block}

\begin{alertblock}{Alert blok}
\verb|\begin{alertblock}{Alert blok}|\\
\end{alertblock}

\begin{exampleblock}{}
\verb|\begin{exampleblock}{}|\\
\end{exampleblock}

\end{frame}


\begin{frame}[fragile]{Postupné odkrývanie bodov}
\begin{itemize}[<+->]
  \item{odkrývanie bez prechodov}
  \item{preloží sa na N stránok}
\end{itemize}
  \begin{block}<3>{ukážka}
    \verb|\begin{itemize}[<+->]|\\
    \verb|    \item{odkrývanie bez prechodov}|\\
    \verb|    \item{preloží sa na N slidov}|\\
    \verb|\end{itemize}|
  \end{block}

\end{frame}


\begin{frame}[fragile]{Ako ukázať zdrojový program}
\begin{block}{Fragile slide}

\verb|\begin{frame}[fragile]{Ako ukázať zdrojový program}|\\
\verb|\begin{block}{Fragile slide}|\\
\verb|    \begin{verbatim}|\\
\verb|    <obsah>|\\
\verb|    \end{verbatim}|\\
\verb|\end{block}|\\
\verb|\end{frame}|\\

\end{block}


\end{frame}



\begin{frame}{Zdroje}
\begin{itemize}
\item \url{http://voho.eu/wiki/latex-beamer/}
\end{itemize}

\end{frame}

\end{document}


